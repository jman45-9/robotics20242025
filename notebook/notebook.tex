\documentclass[12pt]{article}

\usepackage{graphicx}

\title{47510W Phoenix Horizon Programming}
\date{2024-2025}
\author{}

\graphicspath { {./figures} }

\begin{document}
 \maketitle
 \tableofcontents
 \newpage

 \section{PID Controller Basics}
    A PID Controller (Proportional–Integral–Derivative Controller) is a control system that brings the system softly to a target, preventing overshoots and making for a more accurate program. PID controllers are used within our program to make more accurate turns and allow fine control over the amount the robot moves.

\subsection{How Does a PID Controller Work?} \label{howPID}
    Each term, $P$ $I$ $D$, each serve a specific role in creating and controlling the input which is used to reach the target. By adding up the values of the terms we get the input. We will analyze the function of each term individually.

\paragraph{P Term}
    The $P$ (Proportional) increases the input proportionally to the distance to the target. In other words, the farther you are from your target the larger the $P$ Term is. $P$ is the error, or distance from the target. We calculate the value of $P$ with the following formula.

    $$ P = T - M$$
    \begin{center}\em{Where T is target and M is the measured current value}\end{center}

    The $P$ Term is the primary way of tuning the PID and is always the first step. The specifics of tuning and how we tune each term will be discussed in a later section.

\paragraph{I Term}
    The $I$ (Integral) is the cumulative error. The $I$ term increases as the program runs and increases faster if farther away from the target. Which allows the controller to power over an obstacle, most often friction close to the target. We calculate the value of $I$ with the following formula.

    $$ I = I_0 + P$$
    \begin{center}\em{Where $I_0$ is the I term from the previous cycle, and P is the current error(The P term)}\end{center}

\paragraph{D Term}
    The $D$ (Derivative) is how fast the error is changing. The D term helps bring the controller out of a oscillating state caused by repeated overshoots, as are often induced by the $P$ term. We calculate the $D$ term with the following formula.

    $$D = P - P_0$$
    \begin{center}\em{Where P is the error(P term) and $P_0$ is the previous error (last P Term)}\end{center}

\paragraph{Putting It Together}
    To the get the input that we feed to our system, usually a motor for our purposes, we add together our three terms. Each of which multiplied by its tuning information. Tuning specifics will be covered in the next section. Our input formula is as follows.

    $$ R = P \cdot K_P + I \cdot K_I + D \cdot K_D$$
    \begin{center}
        \em{Where R is the output(or return), P I and D are the calculated values, and $K_P$ $K_I$ and $K_D$ are the tuning values}
    \end{center}

\subsection{Tuning A PID}
    For a PID to properly reach the target it must be tuned. Inadequate tuning can lead to failure to reach the target or aggressive oscillations. We tune a PID by changing tuning parameters, these parameters are called $K_P$ $K_I$ and $K_D$. The first step is to set all tuning parameters to 0. By setting a parameter to zero we can cut out their influence for the time being. As in the previous section we will look at the tuning of each term individually.

\paragraph{Tuning the $K_P$ Term}
    The first step to tuning a PID is to set the $K_P$ term. The first step to setting $K_P$ is to increases it until we get an overshoot and a settle either on the target or settles into an undershoot. If it settles on the target we are done with tuning and PID should be ready to use. If we settle into an undershoot we must move onto setting the $K_I$ term. If we overshoot or enter oscillations without settling we need to decrees the $K_P$ term.

\paragraph{Tuning the $K_I$ Term}
    The purpose of changing the $K_I$ term is to fix steady state error after the control settles. As a result of the behavior of the $I$ term to increases over time, it can add input to make the final push to the target even after the distance is not great enough for the $P$ term to have an effect. The $K_I$ does not need to be changed in all application. The $K_I$ term should be increased with caution because increasing it even by small amounts can introduce instability and overshoot.

\paragraph{Tuning the $K_D$ Term}
    The purpose of the $K_D$ term is to control oscillations in the system. We should change $K_D$ when the system has small but unacceptable oscillations. If the oscillations are large, it points to error in the $K_P$ term. Changing $K_D$ is often unnecessary in the context of robotics. We should change $K_D$ to refine out small errors when working with more precise systems such as driving and tuning. However systems which do not need to be super precise, like the speed of a flywheel, we can forgo tuning this term. Tuning of $K_D$ is a final refinement step to add extra stability or to counteract instability added by tuning $K_I$.

\newpage
\section{PID in Code}
    When using a PID for an actual application we can use the information described above to create a foundation, but we also need to add some extra parts to help with the imperfections of the real world. The change that must be made is to tell the program when it is ``Close Enough'' and it can stop. There is also the problem of feeding the input to the application. First we will look at the parts which are similar to the PID theory above.

\subsection{Initializing Variables and Setting Up}
    The first step to creating any new code system is to create an object (See the ``Overview of Object Oriented Programming'' to get information on this concept), and create variables we need. For now we will only create variables related to the theory and not the modifications required for a code implementation. This is all done in the header file (Also see the ``Overview of Programming Concepts'' for more information). So far, our code looks like the following.

    \begin{verbatim}
        in pid.hpp:

        class PID_OBJ
        {
            public:
                double p;
                double i;
                double d;
                double kp;
                double ki;
                double kd;

                double lastErr;

                PID_OBJ(double kp, double ki, double kd);

                double pidCalc(double target, double current);
                void reset(double target);
        };
    \end{verbatim}

    Let's break down that chunk of code a piece at a time. The first thing we see is the class declaration and the public keyword. These are explained in depth in the ``Overview of Object Oriented Programming''. For our purposes of explaining PID, they are unimportant. Then we see a list of variables all declared as floating point(decimal) values with double precision(hence the name 'double'). After that we declare 3 functions which we will look at in a moment. First, however, let us analyze the variables.

   \textbf{Variables.} We can see that we have many familiar variables which are reminiscent of the equations in the previous section. We also have a new variable \verb|double lastErr| which will keep track of what the error was during the last cycle of the control loop. If you recall from the section of the $D$ term, having access to the last error is important to the calculation of the term.

    \textbf{Functions.} Now we will look at the functions we have declared. The first function \verb|PID_OBJ| is a constructor and is called on the creation of a new PID. It takes three arguments, \verb|kp ki kd|, which should be familiar from both the section on tuning and our variables above. These arguments are how we will tune our system. The \verb|pidCalc| function does the operations described in section \ref{howPID}. Finlay the reset function allows us to reset the terms without creating a new PID. This is important between tasks to prevent interference. First we will look at the constructor and reset functions, then we will look at the \verb|pidCalc| function.

\subsection{Constructor and Reset Functions}
    The constructor and reset functions serve to initialize the PID before an external control loop. We will first look at the constructor.

    \textbf{Constructor.} The constructor serves to build a new PID object and set tuning information. The code within the constructor is as follows.

    \begin{verbatim}
    in pid.cpp:

        this->kp = kp;
        this->ki = ki;
        this->kd = kd;
    \end{verbatim}

    We are simply setting the passed values we saw in the header to internal values in the object which we can access later to do calculations with. This allows us to tune once. However, to use the PID multiple times we need to look at the rest function.

    \textbf{Reset.} The reset function sets the values of the PID to be ready for operation. It is absolutely necessary that this function is called before attempting to use the PID. We write the reset function like so.

    \begin{verbatim}
    in pid.cpp:

        this->p = 0;
        this->i = 0;
        this->d = 0;
        this->lastErr = target;
    \end{verbatim}

    We set each term to zero so the calculation has a clean slate. We set \verb|lastErr| to our target value because before the robot has moved, we can assume that the error(distance to the target) is equal to the target. This code could also be placed in the constructor, but this creates a ``Disposable PID'' which can only be used once. By moving the reset code into it's own function, we can reuse our PID.

\subsection{pidCalc Function}

    The \verb|pidCalc| function is the heart of our PID code. It calculates the P I and D terms and adds them together. Later we will also add code to this function to deal with the inconsistency of the robot. First we will look at calculating our terms. It is worth noting that target and current are passed in from the control outside the function, which will discuss later.
    \begin{verbatim}
        this->p = target - current;
        this->i += this->p;
        this->d = this->p - this->lastErr;
        this->lastErr = this->p;
    \end{verbatim}

    We see that we calculate the terms according to the formulas above just with different names. The symbol ``+='' after 'i' tells us that we should add the value on the right to the value on the left. We set \verb|lastErr| after we calculate 'd' because it is no longer relevant to this cycle and it will be equal to 'p' on the next cycle.

    After this we have a line to put the parts together.
    \begin{verbatim}
     return (this->p*this->kp)
        +(this->i*this->ki)+(this->d*this->kd);
    \end{verbatim}

    This line instructs the function to add the terms together after multiplying with their tuning information and return the sum as a result. We now have the foundations of a PID. Before we can implement it, we need to add code to deal with the real world.

\subsection{Dealing With Real World Inaccuracy}
    Sadly, the robot is not perfect and is not able to make zero tolerance turns. Our sensors also have some Inaccuracy that makes setting exact values impossible. Due to this fact we need to add code to call close enough when we are within a satisfactory zone. The first step is to add the following variables to our PID object.

    \begin{verbatim}
        double acceptableErr;
        bool isSettled;
        int acceptableCycles;
    \end{verbatim}

    Let's look at what each of these variables will do. The \verb|acceptableErr| variable will be set in the constructor and tells the PID how close is close enough. Like the tuning values, this needs to be set on a case by case basis for each application, however as a general rule values closer to zero will have more accurate results but require more precision. Values to close to zero may never settle. The \verb|isSettled| is a true or false value that will tell the control loop that we have settled and to move on. In a drive train context, the motors will be given stop commands after isSettled is set to true. Finally the \verb|acceptableCycles| tracks how many cycles we have been in range for.

    We start with checking if we are in an acceptable range.

    \begin{verbatim}
        if(fabs(p) <= acceptableErr)
    \end{verbatim}

    In this line we are saying that if the absolute value of our error is within the acceptable range we can move on. Otherwise, we go to the else condition which sets the acceptableCycles to zero. This else conditions means that we need to be in an acceptable range for a continuous span of time as opposed to passing through it a set number of times. The next line is the next step when we are within acceptable range.

    \begin{verbatim}
      if(this->acceptableCycles++ == 15)
        {
            this->isSettled = 1;
        }
    \end{verbatim}

    These lines add one to the acceptableCycles variable (reflecting the fact we just passed the acceptableCycles check) and then checks if we have at least 15 acceptable cycles. If we have reached this target we set is settled to True (represented by 1), otherwise we do nothing. Note that there is nothing special about 15. Any number will work there but larger numbers will be more accurate but take longer to settle.

    By putting these parts together we get code that is ready to handle the imperfection of the real world. Our full pidCalc function is as follows. (on next page)

    \begin{verbatim}
        in pid.cpp:

        this->p = target - current;
        this->i += this->p;
        this->d = this->p - this->lastErr;
        this->lastErr = this->p;

        if(fabs(p) <= acceptableErr)
        {
            if(this->acceptableCycles++ == 15)
            {
                this->isSettled = 1;
            }
        } else {
            acceptableCycles = 0;
        }
        return (this->p*this->kp)
            +(this->i*this->ki)+(this->d*this->kd);
    \end{verbatim}

    To use this function to run a system we need to create a control loop. We will use the autonomous tuning code as an example of this.

\subsection{PID Control Loop Example}
\paragraph{What does the control loop do?}
    By it's self the PID object is not able to affect the robot in any way. To put our code to work we need to add a control loop. This will take the outputs from the PID and give commands to the robot. We will show an example using autonomous turning code as this is a common use for PID.

\paragraph{Turning Example}

    We should first establish how we will track our current value. On the robot we have a sensor called the inertial sensor. This sensor tracks a number of values but the one we care about is rotation. Rotation tracks the how far we have rotated left or right in an angle. We will pass a value \verb|deg| to our turn function which will instruct the robot on how far it is to turn.

    Our first step is to calculate our starting value and reset the PID. We calculate our target with the following formula.
    $$ T = d + m $$
    \begin{center}
     \em{Where T is target, d is the value deg that we pass, and m is the measured current value}
    \end{center}

    After this we reset our PID, passing the target to the reset function. Our first two lines look as follows.

    \begin{verbatim}
        double target = deg + this->readInertialDeg();
        this->turPID.reset(target);
    \end{verbatim}

    Note that turPID is the PID that is set up with the tuning information regarding turning. Structure of the autonomous program will be discussed later.

    Next we create a loop which will calculate the current PID value and feed it to the motors. The code for that looks as follows.

    \begin{verbatim}
        while(true) {
            double calcVal = this->turPID.pidCalc(target,
                this->readInertialDeg());

            this->driveRight(calcVal);
            this->driveLeft(-calcVal);
    \end{verbatim}

    Note that the value we feed to the left motors is the opposite of that fed to the right motors. This is done so the robot will actually turn as opposed to going straight. We pass the value from the inertial sensor as the current mesured value to the pidCalc function of the turPID (turn PID).

    Next we check if the PID has settled with a quick if statement, which is the following.
    \begin{verbatim}
     if(this->turPID.isSettled)
        {
            this->brake();
            return;
        }
    \end{verbatim}

    If the PID has settled the program tells the robot to brake and stop turning. After braking the loop ends and the robot has turned to the correct positon. This concludes the section on the PID and we will now move onto odometry.


\pagebreak
\section{Basics of Odometry}
\subsection{What is Odometry}
    Odomotry allows the robot to keep track of where it is on the field. In this imperfection, the robot keeps track of whre it is relative to it's starting positon. Odometry works by tracking the rotation of specific wheels and the heading of the robot.
    %should add odom wheel image here later :)

    By using the data from the wheels and sensors we are able to
    With odomotry we are able to tell the robot to go to a point as opposed to manually setting turns and movement. This allows for more percise programing that can remove errors in real applications.

\subsection{Odomotry Sensors}
    Working with odomotry requires two kinds sensors. One to track the robot heading, the inertial sensor, and one to track the positon of the odom wheels, the rotation sensor. We could also use a motor encoder, a comparison table of the two follows this.
    \begin{center}
        \begin{tabular}{c|c}
            Rotation Sensor & Motor Encoder \\
            \hline
            One Smart Motor Point & Two 3 wire ports \\
            More Percise & Less Percise \\
            Less Structure & More structure \\
            Center Screw Pivot & Must have makeshift pivot \\
        \end{tabular}

        We decided to use the rotation sensor because we benifit from the increased percision and better pivot.
    \end{center}

\subsection{Odom Wheels}
We have decided to use 2.75in wheels for odomotry. We chose this size because it is the smallest size we have. It is benifical to use the smallest wheels avalible because the take of the least space. <Add spacing information>

\pagebreak
\subsection{Determining Field Position}
The first step in dertermining the change in the field postion is determining how much the robot heading changed. We use a continuous heading approach meaning that the heading does not reset after completing a rotation. Turing clockwise is repeated by a negitive number and turing counterclockwise is represented by a positive number. We find the change in heading with the following formula.
$$
    \Delta \theta = \theta_1-\theta_0
$$
\textit{Where $theta_1$ is the current heading, $\theta_0$ was the last heading, and $\Delta \theta$ is the change in heading}

Next we must get the change on both authon wheels by using our rotation sensors. We can ues the following formulas.

\textit{Let $degToIn(\theta)$ is a function which converts a value in degrees to a value in inches using the wheel diameter. (Wheel diameter is asuumed in these formulas so is not listed.)}

See \ref{app_degToIn} for more information about degToIn including a formal derivation.
$$
    \Delta x = degToIn(R_{x_1} - R_{x_0})
$$
\textit{Where $\Delta x$ is the change in the $x$ direction. $R_{x_1}$ is the curent positon and $R_{x_0}$ is the last position of the wheel}
$$
    \Delta y = degToIn(R_{y_1} - R_{y_0})
$$
\textit{Where $\Delta y$ is the change in the $y$ direction. $R_{y_1}$ is the curent positon and $R_{y_0}$ is the last postion of the wheel}

\newpage
\appendix
\section{Mathematical Derivations for Odomotry Functions}
\subsection{Deriving the degToIn Function} \label{app_degToIn}
    The degToIn function uses basic circle geometry to convert the degress a wheel rotates into the distance the wheel moves in inches using the wheel diameter. This function hinges on the principle that a wheel moves a distance equivlent to it's circumfrence for each full rotation. We define circumfrence with the equation.
    $$ C = d\pi $$
    \center\textit{Where $C$ is the circumfrence and $d$ is the diameter of the circle.}

    \raggedright
    This formula gives us the distance that the wheel moves if the wheel moves one rotation at a time. Sadly, though, in real life the wheel moves in fractional increments. We can represent this fractional rotation by multiplying the circumfrence by the same fraction. We find this fraction by dividing the mesured valued by the ammount of one rotation, in this case 360 degrees. We can use this fraction of a rotation to get the distance by multiplying it by the wheel circumfrence. So we use the following formula.
    $$ D = \frac{\theta^\circ}{360} \cdot C $$
    \center\textit{Where $D$ is the distance the wheel moved, $\theta^\circ$ is the degrees the wheel rotates, and $C$ is the circumfrence of the wheel.}

    \raggedright
    Finnaly we add this fumction into our code. We use M PI to put $\pi$ into the code. The function goes as follows.

    \begin{verbatim}
     double degToIn(double deg, double wheelDiam)
     {
         return (deg * (M_PI / 360.0)) * (wheelDiam);
     }
    \end{verbatim}

    \textit{Q.E.F}

\end{document}
